\section{Problemstellung}
\label{sec:problemstellung}

\sa{Schreibt das aus \cref{sec:sota} und \cref{sec:researchgap} am Ende zusammen, also im Grunde eine Kurzzusammenfassung von diesen beiden Kapiteln}

Die Simulation von Verkehrsflüssen bereitet aufgrund des unsicheren Faktors Mensch, der als Hauptenscheider nur durch Wahlmöglichkeiten mit vorgegebenen unterschiedlich hohen Wahrscheinlichkeiten modelliert werden kann, noch immer Probleme.

Nagel und Schreckenberg ist es 1992 gelungen\footnote{\cite{na-sch}}, mit einfachen Regeln das mikroskopische Verhalten jedes Fahrzeugführers so abzubilden, dass sich die makroskopische Sicht auf den Verkehrsfluss realistisch darstellte.
Erstmals konnte das Entstehen von \enquote{Stau aus dem Nichts} und das Vorhandensein von \enquote{Stauwellen}, die sich rückwärts durch einen solchen Stau bewegen, simulatorisch dargestellt werden.

Es folgte eine Ausdehnung dieses Modells auf Mehrspurigkeit.\footnote{\cite{multi-lane}} 
Hierfür sind Regeln für Spurwechsel nötig.
Diese sind fest vorgegeben und werden von jedem \enquote*{Fahrzeugführer} in einer identischen Situation befolgt.

Dieses Verhalten trifft die Realität aber nur bedingt.
Eine gewisse, wahrscheinlich geringe, Anzahl an Fahrern könnte sich, evtl. aufgrund von Ängsten, Ablenkungen oder auch Fehleinschätzungen, gegen einen Spurwechsel entscheiden, bzw. diesen nicht in Erwägung ziehen.

Diesem Verhalten trägt eine theoretische Modellentwicklung\footnote{\cite{dat-ba}} Rechnung.
Durch individuelle Berechnungen der Tendenz zum Fahrspurwechsel und einem stochastisch basierten Wahlprozess könnte/soll es möglich sein, Verhalten, wie sie oben beschrieben wurden, zu simulieren.


% Bisher verzichtete man auch auf die mikroskopische Modellierung der Entscheidungen des Individuums und betrachtete die makroskopische Sicht des Verhaltens des Gesamtsystems.