\section{State-of-the-Art}
\label{sec:sota}


Den Straßenverkehr theoretisch abzubilden, ist fachübergreifend eine große Herausforderung in der Forschung.
Hier ergeben sich u. a. Aufgaben im Bereich der Mathematik (Routenplanung z. B. durch graphentheoretische Ansätze), Psychologie (Erforschung menschlichen Verhaltens) und nicht zuletzt Informatik (Simulationen).

Die Simulation von Fahrverhalten ist aufgrund des unsicheren Faktors Mensch, der als Haupt"-entscheider nur durch Wahlmöglichkeiten mit vorgegebenen unterschiedlich hohen Wahrscheinlichkeiten modelliert werden kann, ein Bereich in dem seit mehr als sechs Jahrzehnten geforscht wird.

Verschiedene Ansätze, Verkehrsflüsse zu simulieren, die auch unterschiedliche Sichtweisen verfolgen - mikro- makro- und mesoskopisch, wurden in dieser Zeit verfolgt - Strö"-mungs"-dy"-na"-mik, boolsche Simulation, Zellularautomaten. Letztere werden seit etwa 25 Jahren für die Verkehrssimulation verwendet.

Nagel und Schreckenberg war es 1992 gelungen, auf Basis solcher Automaten mit einfachen Regeln das mikroskopische Verhalten jedes Fahrzeugführers so abzubilden, dass sich die makroskopische Sicht auf den Verkehrsfluss als für Autobahnverkehr realistisch darstellte \cite{na-sch}.
Erstmals konnte das Entstehen von \enquote{Stau aus dem Nichts} und das Vorhandensein von \enquote{Stauwellen}, die sich rückwärts durch einen solchen Stau bewegen, simulatorisch dargestellt werden.

Vereinfacht kann eine Fahrspur einer Straße als Aneinanderreihung vieler Zellen gesehen werden und damit durch den Zellularautomaten abgebildet werden. 
Dies führt schließlich zu einer gridbasierten Simulationsumgebung. 
Die Rechenumgebung wurde als eindimensionales Array mit $L$ Zellen definiert. 
Jede der Zellen kann entweder von einem Fahrzeug belegt oder frei sein. 
Eine Simulation in dieser Grid-Welt hat den Vorteil, dass die Erkenntnisse auf die reale Welt skaliert werden können. 
Es wurde eine Zelllänge von 7,5\nolinebreak[4] m angenommen (vgl. \cite[S. 2227]{na-sch}), was ungefähr dem beanspruchten Platz eines Pkw - von Fahrzeugfront zu Fahrzeugfront - in einer Stausituation entspricht (Fahrzeuglänge + Abstand). 
Jedes Fahrzeug hat eine ganzzahlige Geschwindigkeit zwischen null und $v_{max}$.

%Im Jahr 1995 wurde dann mit Hilfe des \enquote{Nagel-Schreckenberg-Modells}, so der heute verwandte und bekannte Name, das gesamte deutsche Autobahnnetz in Echtzeit simuliert. 
%Ebenso wird das Modell im Rahmen des Programmes TRANSIMS zur Simulation des Straßenverkehrs der Schweiz eingesetzt. (vgl. \cite{spahn-da})
1996 folgte eine Ausdehnung des Modells auf Mehrspurigkeit \cite{multi-lane}. 
Das Originalmodell wurde um Regeln, die prüfen, ob ein Spurwechsel vorteilhaft ist und einen sicheren, weil kollisionsfreien, Spurwechsel ermöglichen, erweitert. 
Wenn es die festgelegten Sicherheitskriterien zulassen, erfolgt der Spurwechsel auf die linke Spur immer, da dieser nicht an eine Wahrscheinlichkeit geknüpft ist.
Lediglich um die Überbevölkerung der Überholspur zu vermeiden, wurde eine Wahrscheinlichkeit $p_{l2r}$ für das zurückwechseln eingeführt.

Was dieser Modellierung fehlt, ist die Berücksichtigung irregulärer Verhaltensweisen der Fahrer - Unsicherheiten (z. B. Fehleinschätzungen oder Ängste), Ablenkungen (z. B. Gespräche mit anderen Fahrzeuginsassen, Telefonieren, Suche im Handschuhfach oder nach der fallen gelassenen Zigarette). 

Zudem ist die Bremswirkung \enquote{überdimensioniert}. 
Vollbremsungen aus Höchstgeschwindigkeit sind innerhalb kürzester Distanz möglich \cite{acc-free}.
Daraus ergibt sich eine weitere fehlende Komponente der Realität.
Denn auch Unfälle gehören zum Verkehrsgeschehen. \\
Studien zufolge sind drei Viertel der Autofahrer abgelenkt, wenn sie am Steuer sitzen. 
Jeder zehnte Verkehrsunfall in Deutschland wird durch unaufmerksame Autofahrer verursacht. 
In Österreich und der Schweiz geht man, aufgrund der Einordnung in eine eigene Kategorie für diese Art Unfälle, von einer Rate von etwa 30\% der Unfälle mit Personenschäden oder gar Toten aus. (vgl. \cite{dvr-studie})

Laut Statistischem Bundesamt gab es 2016 auf Autobahnen 21193 Unfälle, davon 1435 innerhalb von Baustellen \cite{unf2016}. 
Bei den etwa 13000 km Autobahn \cite{autob2016} entspricht dies durchschnittlich etwa einem Unfall pro 200 km Strecke pro Tag. 
In Wirklichkeit wird es aber Unfallschwerpunkte geben, die diesen Durchschnittswert überschreiten.

Es gibt inzwischen Ansätze, die das mikroskopische Verhalten der Fahrzeugführer aus Realdaten  stochastisch modelliert haben \cite{stoch-carfollow}. 
In den aggregierten Daten wurde eine Laplace-Verteilung der Beschleunigungswerte erkannt. 
Das daraus entwickelte Modell wurde getestet. 
Die Reproduktion der Sicherheitsparameter, wie Zeit bis zur Kollision (TTC, time to collision) und der Bremsrate um einen Unfall zu verhindern (DRAC, deceleration rate to avoid crash) gelang. 
Allerdings konnte das Fundamentaldiagramm nicht nachgestellt werden.

\cite{multi-fuzzy} integriert den Ansatz der Mehrspurigkeit in einem Multiagenten-Fuzzy-System. 
Die Spuren werden als Vereinigung mehrerer miteinander kommunizierender einspuriger Zellularautomaten (\enquote{continuous cellular automata}) beschrieben. 
Diese Kommunikation zwischen den Spuren beschränkt sich auf die Sicherheitskriterien, wenn ein Fahrzeug die Spur wechseln möchte. 
Intentionen zum Wechsel zwischen den Spuren werden durch fortwährend aktualisierte \enquote{Stresslevel} je Fahrzeug beeinflusst, die Wahrscheinlichkeiten dafür durch einen Bernoulli-Prozess berechnet.
Dieser Ansatz arbeitet mit Gridzellen und achtet auf Kollisionsfreiheit.

In \cite{dat-ba} wurde mit dem \enquote{Social-Force-Vehicle-Modell} ein neuer Vorschlag für die Simulation von Spurwechseln entwickelt. 
Das Modell orientiert sich bei den Fahrzeugbewegungen an den Grundlagen des Nagel-Schreckenberg-Modells, vermeidet Unfälle aber in erster Linie nicht durch Abbremsen, sondern durch Ausweichen, also Spurwechsel. 
Im Gegensatz zur ursprünglichen erdachten Nutzung der \enquote{Social Forces} für Fußgänger, wie z. B. in \cite{soc-for} beschrieben, wird dieses System auf die Gridzellen, wie sie in \cite{na-sch} vorkommen, angewandt und wirken als eine Art \enquote{Anziehungskraft}. 
Die Kräfte der Zellen %, siehe \cref{figure:social-forces}, 
werden für jedes Fahrzeug zu jedem Zeitschritt $t$, da dessen Position zum Zeitpunkt $t + 1$ von der Position im vorhergehenden Zeitschritt abhängt, über eine Exponentialfunktion berechnet.
Von den ein Fahrzeug umgebenden acht Zellen müssen, aufgrund der Bewegungsrichtung, nur die drei betrachtet werden, die sich in Fahrtrichtung befinden.
Die letztendliche Auswahl der Zielzelle für einen \enquote{Impuls zur Spurwahl} wird mittels \enquote{Fitness-proportionate Selection} durchgeführt.
Entgegen des Zwangs zum Ausscheren oder einer fest vorgegebenen Wahrscheinlichkeit wird eine stochastische Simulationsmöglichkeit der Neigung zum Spurwechsel aufgezeigt.





%Bereits seit den 1950er Jahren wurden strömungsdynamische Ansätze für die Simulation von Verkehrsflüssen entwickelt. 
%Außerdem gab es ab den 1980er Jahren auch boolsche Simulationsmodelle, die mit Gitter-Gas-Automaten Flüssigkeiten simulieren konnten.
%Anfang der 1990er wurde von Kai Nagel und Michael Schreckenberg in \cite{na-sch} ein Verfahren vorgestellt, welches Autobahnverkehr basierend auf Zellularautomaten modelliert. 
%
%\begin{quote}
%Ein zellulärer Automat ist eine regelmäßige Annordnung von Zellen. Jede Zelle kann eine endliche Zahl von Werten / Zuständen annehmen und hat eine  begrenzte Zahl von Nachbarzellen, die sie beeinflussen können. Das Muster des gesamten zellulären Automaten ändert sich in einzelnen Schritten, die durch eine Reihe von Übergangsregeln bestimmt werden, die für alle Zellen gelten. (aus \cite{cell-autom})
%\end{quote}
%
%\noindent
%Vereinfacht kann eine Fahrspur einer Straße als Aneinanderreihung vieler solcher Zellen gesehen werden. 
%Dies führt schließlich zu einer gridbasierten Simulationsumgebung. 
%In \cite{na-sch} wird die Rechenumgebung als eindimensionales Array mit $L$ Zellen definiert. 
%Jede der Zellen kann entweder von einem Fahrzeug belegt oder frei sein. 
%Eine Simulation in dieser Grid-Welt hat den Vorteil, dass die Erkenntnisse auf die reale Welt skaliert werden können. 
%Es wird eine Zelllänge von 7,5\nolinebreak[4] m angenommen (vgl. \cite[S. 2227]{na-sch}), was ungefähr dem beanspruchten Platz eines Pkw - von Fahrzeugfront zu Fahrzeugfront - in einer Stausituation entspricht (Fahrzeuglänge + Abstand). 
%Jedes Fahrzeug hat eine ganzzahlige Geschwindigkeit zwischen null und $v_{max}$.
%
%\begin{table}[ht]
%\begin{center}
%\setlength{\tabcolsep}{0.5em} % for the horizontal padding
%{\renewcommand{\arraystretch}{1.2}% for the vertical padding
%\begin{tabular}{| c | c | c |}
%\hline 
%$v^{sim}$ in $\frac{Zellen}{Zeitschritt}$ & $\widehat{=}$ $v^{real}$ in $\frac{m}{s}$ & $=$ in $\frac{km}{h}$ \\ \hline 
%$1$ & 7,5 & 27 \\ \hline
%$2$ & 15 & 54 \\ \hline
%$3$ & 22,5 & 81 \\ \hline
%$4$ & 30 & 108 \\ \hline
%$5$ & 37,5 & 135 \\ \hline
%$6$ & 45 & 162 \\ \hline
%\end{tabular}
%}
%\caption{Umrechnung Geschwindigkeiten Gridwelt $\rightarrow$ reale Welt}
%\end{center}
%\label{tab:umrechnung-zelle-kmh}
%\end{table}
%
%\noindent
%Die beobachtete Durchschittsgeschwindigkeit von 4,5 Zellen/Zeitschritt, der als eine Sekunde angenommen wird, entspricht etwa der einer Geschwindigkeit von 120 km/h (vgl. \cite[S. 2227]{na-sch}). 
%
%In \cite{na-sch} war es erstmals gelungen, Wechselwirkungen zwischen Fahrzeugen im Einspurfall darzustellen und u. a. das Entstehen von Staus, ohne dass ein Grund dafür vorlag, zu modellieren. 
%Mit drei Regeln, die auf jedes Fahrzeug gleichzeitig und gleichermaßen anzuwenden waren, wurde ein realistischer Verkehrsfluss generiert. 
%Die folgenden Regeln, die die Kollisionsfreiheit sichern, sind die Grundlage für die, heute als \enquote{Nagel-Schreckenberg-Modell} bekannte, Modellierung:
%
%\begin{itemize}
%\item \textit{Beschleunigung}: Solange ein Fahrzeug nicht seine max. Geschwindigkeit $v_{max}$ (Zellen/""Zeitschritt) erreicht hat und ein vorausfahrendes Fahrzeug weit genug entfernt ist, erhöht es die Geschwindigkeit um $1$, $v \rightarrow v+1$;
%\item \textit{Abbremsen}: Hat ein Fahrzeug ein anderes Fahrzeug $j$ Zellen vor sich, dann reduziert es die Geschwindigkeit auf $j-1$, $v \rightarrow j-1$;
%\item Zufallsgröße, auch \textit{\enquote*{Trödelwahrscheinlichkeit}}: Mit einer Wahrscheinlichkeit $p$ wird die Geschwindigkeit ($v > 0$) eines Fahrzeuges um $1$ reduziert, $v \rightarrow v-1$
%\end{itemize}
%
%\noindent
%Die Anweisungen werden in der angegebenen Reihenfolge - für alle Fahrzeuge gleichzeitig - ausgeführt und jedes Fahrzeug um die entsprechende Anzahl Zellen weiter gesetzt.
%
%Dehnt man dieses Modell aber auf mehrere Fahrspuren aus, stößt es an seine Grenzen, weil die Modellierung von Überholvorgängen, bzw. genauer gesagt für das \enquote{Ausscheren} und \enquote{Einscheren}, nicht Teil des ursprünglichen Modells ist. 
%\cite{multi-lane} liefert eine \enquote*{multi-lane}-Betrachtung, wobei die ursprünglichen drei Regeln durch weitere ergänzt wurden, die zum einen ein Auffahren des Nachfolgeverkehrs auf den Spurwechsler und zum anderen auch des die Spur wechselnden auf vorausfahrende Fahrzeuge verhindern. \\
%In der Betrachtung, ob die Aktualisierung der Fahrzeuge parallel oder sequentiell erfolgen soll, wurde erkannt, dass dies für das entwickelte Modell nur geringe Unterschiede macht, da die Rate der Spurwechsel mit den festgelegten Regeln eher gering ist. \\
%Gleichzeitig wurde auch beobachtet, dass Fahrzeuge nicht wieder von der Überholspur in die Normalspur wechselten. Dies wurde mit weiteren Regeln und einer Spurwechselwahrscheinlichkeit abgestellt. Eine weitere Kalibrierung des Modells wurde in einer Folgearbeit durchgeführt. \\
%Durch die Konstanthaltung der Anzahl generierter Fahrzeuge und die Veränderung der Systemgröße konnten verschiedene Verkehrsdichten simuliert werden.
%
%\cite{multi-lane} zeigt mehrere Diagramme. Eine Untersuchung von realen Spurwechselvorgängen ergab, dass eine Umkehr der Benutzungshäufigkeit der rechten und linken Spuren bei einem Verkehrsfluss $q_{12}$ von etwa 1200 (Fahrzeugen) pro Stunde auf beiden Spuren erfolgt.
%
%Die Simulation ergab, dass der Punkt, an dem beide Spuren gleichermaßen benutzt werden, deutlich unter dem Punkt liegt, an dem der maximale Verkehrsfluss erreicht wird, siehe \cref{figure:verkehrsfluss-spurnutzung}.
%
%\begin{figure}[hptb]
% \centering
% \includegraphics[width=0.7\textwidth]{verkehrsfluss-spurnutzung}
% \caption[Verkehrsfluss und Spurnutzung als Funktion der Verkehrsdichte]{Verkehrsfluss und Spurnutzung als Funktion der Verkehrsdichte, aus \cite{multi-lane}}
% \label{figure:verkehrsfluss-spurnutzung}
%\end{figure}
%
%\noindent
%Ein weiteres Resultat der Simulation ist eine mehrfache Umkehr der Nutzung der Fahrspuren abhängig von der Verkehrsdichte, wenn andere Parameter konstant gehalten werden, siehe \cref{figure:verkehrsfluss}.
%
%\begin{figure}[hptb]
% \centering
% \includegraphics[width=0.7\textwidth]{verkehrsfluss}
% \caption[Verkehrsfluss als Funktion der durchschnittlichen Verkehrsdichte]{Verkehrsfluss auf den Fahrspuren als Funktion der durchschnittlichen Verkehrsdichte, aus \cite{multi-lane}}
% \label{figure:verkehrsfluss}
%\end{figure}
%
%\noindent
%Die Festlegung einer Wahrscheinlichkeit für das Zurückwechseln von der linken auf die rechte Fahrspur führte zu einer unterschiedlichen Spurwechselfrequenz, siehe \cref{figure:spurwechselfrequenz}.
%
%\begin{figure}[hptb]
% \centering
% \includegraphics[width=0.7\textwidth]{spurwechselfrequenz}
% \caption[Spurwechselfrequenz als Funktion der Verkehrsdichte]{Spurwechselfrequenz als Funktion der Verkehrsdichte für verschiedene Spurwechselwahrscheinlichkeiten, aus \cite{multi-lane}}
% \label{figure:spurwechselfrequenz}
%\end{figure}
%
%\pagebreak
%In \cite{dat-ba} wurde mit dem \enquote{Social-Force-Vehicle-Modell} ein neuer Ansatz für die Simulation von Spurwechseln entwickelt.
%Auch hier wird eine Grid-Umgebung verwendet.
%Im Gegensatz zur ursprünglichen erdachten Nutzung der \enquote{Social Forces} für Fußgänger, wie z. B. in \cite{soc-for} beschrieben, wird dieses System auf die Gridzellen angewandt. Die Kräfte der Zellen, siehe \cref{figure:social-forces}, werden für jedes Fahrzeug zu jedem Zeitschritt $t$ berechnet, da seine Position zum Zeitpunkt $t + 1$ von der Position im vorhergehenden Zeitschritt abhängt.
%Von den ein Fahrzeug umgebenden acht Zellen müssen, aufgrund der Bewegungsrichtung, nur die drei betrachtet werden, die sich in Fahrtrichtung befinden.
%
%\begin{figure}[hptb]
% \centering
% \includegraphics[width=0.7\textwidth]{social-forces}
% \caption[\enquote{Social Forces} für Grid-Zellen]{Vorschlag der \enquote{Social Forces} für Grid-Zellen, aus \cite{dat-ba}}
% \label{figure:social-forces}
%\end{figure}
%
%\noindent
%Die letztendliche Auswahl der Zielzelle wird mittels \enquote{Fitness-proportionate Selection} durchgeführt.
%Entgegen des Zwangs zum Ausscheren und einer fest vorgegebenen Wahrscheinlichkeit wird eine stochastische Simulationsmöglichkeit der Neigung zum Spurwechsel aufgezeigt.