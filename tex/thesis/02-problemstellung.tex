\section{Problemstellung}
\label{sec:problemstellung}







%% sa{Schreibt das aus \cref{sec:sota} und \cref{sec:researchgap} am Ende zusammen, also im Grunde eine Kurzzusammenfassung von diesen beiden Kapiteln}
%
%Die Simulation von Verkehrsflüssen ist aufgrund des unsicheren Faktors Mensch, der als Haupt"-entscheider nur durch Wahlmöglichkeiten mit vorgegebenen unterschiedlich hohen Wahrscheinlichkeiten modelliert werden kann, ein Bereich in dem seit mehr als sechs Jahrzehnten geforscht wird.
%
%Verschiedene Ansätze, Fahrverhalten zu simulieren, wurden in dieser Zeit verfolgt - Strö"-mungs"-dy"-na"-mik, boolsche Simulation, Zellularautomaten. Letztere werden seit etwa 25 Jahren für die Verkehrssimulation verwendet.
%
%Nagel und Schreckenberg ist es 1992 in \cite{na-sch} gelungen, mit einfachen Regeln das mikroskopische Verhalten jedes Fahrzeugführers so abzubilden, dass sich die makroskopische Sicht auf den Verkehrsfluss realistisch darstellte.
%Erstmals konnte das Entstehen von \enquote{Stau aus dem Nichts} und das Vorhandensein von \enquote{Stauwellen}, die sich rückwärts durch einen solchen Stau bewegen, simulatorisch dargestellt werden.
%
%1995 gelang es mit Hilfe des \enquote{Nagel-Schreckenberg-Modells}, so der heute verwandte und bekannte Name, das gesamte deutsche Autobahnnetz in Echtzeit zu simulieren. 
%Ebenso wird das Modell im Rahmen des Programmes TRANSIMS zur Simulation des Straßenverkehrs der Schweiz eingesetzt. (vgl. \cite{spahn-da})
%
%1996 folgte in \cite{multi-lane} eine Ausdehnung des Modells auf Mehrspurigkeit.  
%Hierfür sind Regeln für Spurwechsel nötig.
%Diese sind fest vorgegeben und werden von jedem \enquote*{Fahrzeugführer} in einer vergleichbaren oder auch nur ähnlichen Situation befolgt.
%
%Dieses simulatorische Verhalten trifft die Realität aber nur bedingt.
%Eine gewisse Anzahl an Fahrern könnte sich, vielleicht aufgrund von Ängsten, Ablenkungen oder auch Fehleinschätzungen, gegen einen Spurwechsel entscheiden, bzw. diesen nicht in Erwägung ziehen.
%Studien zufolge sind drei Viertel der Autofahrer abgelenkt, wenn sie am Steuer sitzen. 
%Jeder zehnte Verkehrsunfall in Deutschland wird durch unaufmerksame Autofahrer verursacht. 
%In Österreich und der Schweiz geht man, aufgrund der Einordnung in eine eigene Kategorie für diese Art Unfälle, von einer Rate von etwa 30\% der Unfälle mit Personenschäden oder gar Toten aus. (vgl. \cite{dvr-studie})
%
%Dieser Verhaltensweise trägt eine theoretische Modellentwicklung in \cite{dat-ba} aus dem Jahr 2017 Rechnung.
%Durch individuelle Berechnungen einer Tendenz zum Fahrspurwechsel, in die die Position und Geschwindigkeit vorausfahrender und nachfolgender Fahrzeuge mit eingehen, und einem stochastisch basierten Wahlprozess könnte es möglich sein, Verhalten, wie sie oben beschrieben wurden, zu simulieren.
%Unterschiedliche Voraussetzungen des Verkehrsgeschehens führen zu unterschiedlichen Wahrscheinlichkeiten und damit zu verschieden hohen Chancen einen Spurwechsel \textit{nicht} durchzuführen.
%
%Es gilt zu klären, ob das theoretische Modell praktisch funktioniert und ob es den Verkehrsfluss realistisch oder sogar realistischer als bisherige Ansätze darstellen kann.