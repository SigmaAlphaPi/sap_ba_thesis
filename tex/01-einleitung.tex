\section{Einleitung}
\label{sec:einleitung}

Als Verkehr kann eine Vielzahl verschiedener Arten der Fortbewegung, des Transports bezeichnet werden. 
Dazu dienen unterschiedliche Verkehrsmittel, mit denen sich an Land, auf dem Wasser oder in der Luft fortbewegt werden kann.
Landbasiert unterscheidet man wiederum in \mbox{straßen-/}wege- oder schienenbasiert.

Um Straßen so auszulegen, dass sie auch noch prognostizierten Verkehr bewältigen können, bedarf es geeigneter Modelle und Simulationsmethoden.
Bereits in der ersten Hälfte des  vergangenen Jahrhunderts gab es Bestrebungen, Verkehrsmuster zu erklären und nachzubilden.

Am Anfang der 1990er Jahre gelang es Kai Nagel und Michael Schreckenberg mit einfachen Regeln das Verhalten eines einzelnen Fahrzeug(führer)s so abzubilden, dass die Interaktion mehrerer dieser Fahrzeuge ein realistisches Verkehrsbild zeigte.
Das damals erstellte Modell trägt den Namen seiner Entdecker, \enquote{Nagel-Schreckenberg-Modell}, und wurde u.a. dafür genutzt, das gesamte deutsche Autobahnnetz in Echtzeit zu simulieren.

Das Modell setzt Homogenität der Verkehrsteilnehmer voraus. 
Dies stellt sich natürlich in der Realität vollkommen anders dar.
Vom \enquote{Schleicher auf der linken Spur} bis hin zum \enquote{Verkehrsrowdy} gibt es eine Vielzahl von Autofahrertypen.
Das Konzept der Agentenprogrammierung erlaubt jetzt, anstatt homogenen Verkehrs unterschiedliche Verkehrssubjekte darzustellen.




\subsection{Gliederung der Arbeit}
\label{sec:gliederung}

Diese Bachelorarbeit besteht aus insgesamt \sa{own: finale Anzahl eintragen} Kapiteln.
Das \cref{sec:sota} enthält einige Arbeiten, die Historie und Stand der Technik auf dem behandelten Gebiet wiederspiegeln.
\cref{sec:simulationsumgebung} beschäftigt sich mit der Thematik Multiagentensysteme und stellt die Simulationsumgebung und das Agenten-Framework vor.
Dem Entwicklungsprozess der das Verhalten des Nagel-Schreckenberg-Modells (NaSch-Modell) als Agentenversion sowie dessen Test widmet sich \cref{sec:realisierung}.
Im darauf folgenden Kapitel werden Erweiterungsideen für eine Mehrspurversion gegeben.


 

 
%Hierfür sind Regeln für Spurwechsel nötig.
%Diese sind fest vorgegeben und werden von jedem \enquote*{Fahrzeugführer} in einer vergleichbaren oder auch nur ähnlichen Situation befolgt.
%
%Dieses simulatorische Verhalten trifft die Realität aber nur bedingt.
%Eine gewisse Anzahl an Fahrern könnte sich, vielleicht aufgrund von Ängsten, Ablenkungen oder auch Fehleinschätzungen, gegen einen Spurwechsel entscheiden, bzw. diesen nicht in Erwägung ziehen.
%Studien zufolge sind drei Viertel der Autofahrer abgelenkt, wenn sie am Steuer sitzen. 
%Jeder zehnte Verkehrsunfall in Deutschland wird durch unaufmerksame Autofahrer verursacht. 
%In Österreich und der Schweiz geht man, aufgrund der Einordnung in eine eigene Kategorie für diese Art Unfälle, von einer Rate von etwa 30\% der Unfälle mit Personenschäden oder gar Toten aus. (vgl. \cite{dvr-studie})
%
%Dieser Verhaltensweise trägt eine theoretische Modellentwicklung in \cite{dat-ba} aus dem Jahr 2017 Rechnung.
%Durch individuelle Berechnungen einer Tendenz zum Fahrspurwechsel, in die die Position und Geschwindigkeit vorausfahrender und nachfolgender Fahrzeuge mit eingehen, und einem stochastisch basierten Wahlprozess könnte es möglich sein, Verhalten, wie sie oben beschrieben wurden, zu simulieren.
%Unterschiedliche Voraussetzungen des Verkehrsgeschehens führen zu unterschiedlichen Wahrscheinlichkeiten und damit zu verschieden hohen Chancen einen Spurwechsel \textit{nicht} durchzuführen.
%
%Es gilt zu klären, ob das theoretische Modell praktisch funktioniert und ob es den Verkehrsfluss realistisch oder sogar realistischer als bisherige Ansätze darstellen kann.