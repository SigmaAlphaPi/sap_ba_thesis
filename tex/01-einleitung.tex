\section{Einleitung}\sa{Leerzeilen entfernen und Einrückung am Anfang der Absätze entfernen}
\label{sec:einleitung}

Mit dem Begriff Verkehr kann eine Vielzahl verschiedener Arten der Fortbewegung oder des Transports bezeichnet werden. 
Dazu dienen unterschiedliche Verkehrsmittel, mit denen sich an Land, auf dem Wasser oder in der Luft fortbewegt werden kann.
An Land unterscheidet man wiederum in \mbox{straßen-/}wege- oder schienenbasierten Verkehr.

Um Straßen so auszulegen, dass sie auch noch prognostizierten Verkehr bewältigen können, bedarf es geeigneter Modelle und Simulationsmethoden.
Bereits in der ersten Hälfte des  vergangenen Jahrhunderts gab es Bestrebungen, Verkehrsmuster zu erklären und nachzubilden.

Am Anfang der 1990er Jahre gelang es Kai Nagel und Michael Schreckenberg mit einfachen mathematischen Regeln das Verhalten eines einzelnen Fahrzeug(führer)s so abzubilden, dass die Interaktion mehrerer dieser Fahrzeuge ein realistisches Verkehrsbild zeigte.
Das damals erstellte Modell trägt inzwischen den Namen seiner Entdecker, das \enquote{Nagel-Schreckenberg-Modell}, und wurde u.a. dafür genutzt, das gesamte deutsche Autobahnnetz in Echtzeit zu simulieren.

Das Modell setzt allerdings Homogenität der Verkehrsteilnehmer voraus. 
Dies stellt sich natürlich in der Realität vollkommen anders dar.
Vom \enquote{Schleicher auf der linken Spur} bis hin zum \enquote{Verkehrsrowdy} gibt es eine Vielzahl von Autofahrertypen und von der \enquote{Ente} bis zum Porsche die unterschiedlichsten Fahrzeuge.
Das Konzept der Agentenprogrammierung erlaubt jetzt, anstatt homogenen Verkehrs eine große Vielfalt unterschiedlicher Verkehrssubjekte (und -objekte) darzustellen und diese miteinander agieren zu lassen.

Mit dieser Arbeit soll gezeigt werden, ob sich das Multiagentenframework \enquote{LightJason} dazu eignet, Verkehrsverhalten nachzubilden.



\subsection{Gliederung der Arbeit}
\label{sec:gliederung}

Diese Bachelorarbeit besteht aus insgesamt \sa{own: finale Anzahl eintragen} Kapiteln.
Das \cref{sec:sota} enthält einige Arbeiten, die Historie und Stand der Technik auf dem behandelten Gebiet wiederspiegeln.
\cref{sec:simulationsumgebung} beschäftigt sich mit der Thematik Multiagentensysteme und stellt die Simulationsumgebung und das Agenten-Framework vor.
Dem Entwicklungsprozess, der das Verhalten des Nagel-Schreckenberg-Modells (NaSch-Modell) als Agentenversion als Ergebnis hat, sowie dessen Test widmet sich \cref{sec:realisierung}.
Im darauf folgenden Kapitel werden Erweiterungsideen für eine Mehrspurversion gegeben.

