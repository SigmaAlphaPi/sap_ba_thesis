\section{Einleitung}
\label{sec:einleitung}


Als Verkehr kann eine Vielzahl verschiedener Arten der Fortbewegung bezeichnet werden. 
Dazu dienen unterschiedliche Verkehrsmittel, mit denen sich an Land, auf dem Wasser oder in der Luft fortbewegt werden kann.
Landbasiert unterscheidet man wiederum in straßen-/wege- oder schienenbasiert.






 
%Hierfür sind Regeln für Spurwechsel nötig.
%Diese sind fest vorgegeben und werden von jedem \enquote*{Fahrzeugführer} in einer vergleichbaren oder auch nur ähnlichen Situation befolgt.
%
%Dieses simulatorische Verhalten trifft die Realität aber nur bedingt.
%Eine gewisse Anzahl an Fahrern könnte sich, vielleicht aufgrund von Ängsten, Ablenkungen oder auch Fehleinschätzungen, gegen einen Spurwechsel entscheiden, bzw. diesen nicht in Erwägung ziehen.
%Studien zufolge sind drei Viertel der Autofahrer abgelenkt, wenn sie am Steuer sitzen. 
%Jeder zehnte Verkehrsunfall in Deutschland wird durch unaufmerksame Autofahrer verursacht. 
%In Österreich und der Schweiz geht man, aufgrund der Einordnung in eine eigene Kategorie für diese Art Unfälle, von einer Rate von etwa 30\% der Unfälle mit Personenschäden oder gar Toten aus. (vgl. \cite{dvr-studie})
%
%Dieser Verhaltensweise trägt eine theoretische Modellentwicklung in \cite{dat-ba} aus dem Jahr 2017 Rechnung.
%Durch individuelle Berechnungen einer Tendenz zum Fahrspurwechsel, in die die Position und Geschwindigkeit vorausfahrender und nachfolgender Fahrzeuge mit eingehen, und einem stochastisch basierten Wahlprozess könnte es möglich sein, Verhalten, wie sie oben beschrieben wurden, zu simulieren.
%Unterschiedliche Voraussetzungen des Verkehrsgeschehens führen zu unterschiedlichen Wahrscheinlichkeiten und damit zu verschieden hohen Chancen einen Spurwechsel \textit{nicht} durchzuführen.
%
%Es gilt zu klären, ob das theoretische Modell praktisch funktioniert und ob es den Verkehrsfluss realistisch oder sogar realistischer als bisherige Ansätze darstellen kann.