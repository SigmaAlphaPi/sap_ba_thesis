[ZU FAZIT]

Wird as Nagel-Schreckenberg-Modell zuk�nftig noch von Bedeutung sein? 
Viele Entwicklungen im Automobilbereich gehen hin in Richtung hochassitentes oder gar autonomes Fahren.

bla bla bla Fehleinsch�tzungen k�nnen auch dem Rechner passieren bla bla bla

Sollten diese Systeme zu 100% sicher funktionieren, kann man den Faktor Mensch, der meist die Unsicherheiten in das System Verkehr einbringt, aus der Gleichung entfernen.
Wichtig w�re dann aber auch die Kommunikation der Fahrzeuge untereinander.

Von diesem Punkt ist die Technik aktuell noch weit entfernt, siehe [www.sueddeutsche.de/auto/autonomes-fahren-der-entwicklungsaufwand-bei-selbstfahrenden-autos-ist-riesig-1.3838094], sodass der Mensch hinter dem Lenkrad wohl noch einige Zeit erhalten bleiben wird \dots und sei es als Backup bei Fehleinsch�tzungen der Technik.

https://www.welt.de/wirtschaft/article168550776/Fuer-autonomes-Fahren-gibt-es-ein-grosses-Hindernis.html



\subsection{Zuf�lligkeit des Aus- und Einscherens}

Um unterschiedliche Verhalten beim �berholen im Autobahnverkehr abbilden zu k�nnen, ben�tigt man einen Mechanismus, Wahrscheinlichkeiten f�r eine der Handlungen zu generieren.
\\
Hier scheint die Sigmoid-Funktion $ f(x) = \frac{1}{1 + e^{-x}} $ ein geeigneter Kandidat. 
Ihr Definitionsbereich sind alle $ R $ und ihr Wertebereich liegt zwischen 0 und 1.
\\
Die Grundfunktion steigt im Intervall -5, 5 ziemlich rasch von nahe 0 auf nahe 1.
Dies kann durch Streckung der Funktion etwas abgemildert werden.
So hat z.B. $ \frac{1}{1 + e^{-\frac{x}{2}}} $ einen nutzbaren Intervall zwischen -10, 10.

F�r das �berholen wurden die folgende Zusammenh�nge erkannt:
\begin{itemize}
    \itemsep0em
    \item Abstand $ \uparrow $, �berholbed�rfnis $ \downarrow $ und umgekehrt
    \item relative Geschwindigkeit $ \uparrow $, �berholbed�rfnis $ \uparrow $ und umgekehrt
\end{itemize}

Es galt sinnvolle Werte zu finden und diese auf den o.g. Intervall zu normieren.

F�r den Abstand wurde die Grenze der [Zone Handlung ... Verweis] - 110 m - und ein Punkt, an dem nach M�glichkeit sp�testens ein �berholvorgang eingeleitet sein sollte - 50 m - gew�hlt.
\\
Dies f�hrt zur Funktion $ f(a) = \frac{1}{1 + e^{-\frac{ \frac{80}{3} - \frac{a}{3} }{2}}} $.

F�r die relative Geschwindigkeit wurden die Grenzen bei 0 und 25 km/h festgelegt.
Ab 0 km/h bzw. darunter ist kein �berholen mehr n�tig.
F�r die obere Grenze wurde das Verhalten auf Landstra�en [https://www.adac.de/_mmm/pdf/fachdossier_ueberholen_auf_landstra�en_68414.pdf, S. 27] auf den Autobahnverkehr �bertragen.
\\
Dies f�hrt zu der Funktion $ f(v_{rel}) = \frac{1}{1 + e^{-\frac{ \frac{4 v_{rel}}{5} - 10 }{2}}} $.

Eine Kombination der beiden Gleichungen 
\\
$ \frac{1}{2}(f(a) + f(v_{rel})) = \frac{1}{2}( \frac{1}{1 + e^{-\frac{ \frac{80}{3} - \frac{a}{3} }{2}}} + \frac{1}{1 + e^{-\frac{ \frac{4 v_{rel}}{5} - 10 }{2}}} ) $ 
\\
liefert die Wahrscheinlichkeit f�r das Ausschehren.

[Grafik WolframAlpha]