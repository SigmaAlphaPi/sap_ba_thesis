\section{Fazit/Ausblick}
\label{sec:fazit-ausblick}






Im Jahr 2015 wurde bei einer Studie der London School of Economics and Political Science (LSE) im Auftrag des Reifenherstellers Goodyear, siehe \cite{fahrertyp}, festgestellt, dass man Autofahrer in sieben Typen einteilen kann.

\begin{itemize}
\itemsep0em
	\item the \enquote{teacher}, der Lehrer: möchte, dass andere Fahrer wissen, was sie falsch gemacht haben und erwartet Anerkennung für seine Anstrenungen andere zu belehren
	\item \enquote{the Know-it-all}, der Besserwisser: denkt, dass er von inkompetenten Schwachköpfen umgeben ist und schreit diese gelegentlich auch schon einmal an, während er sicher in seinem Auto sitzt
	\item \enquote{the Competitor}, der Wettkämpfer: muss vor allen anderen Fahrern sein und reagiert verärgert, wenn sich diesem Ziel jemand entgegen stellt, beschleunigt evtl. falls jemand zu überholen versucht oder schließt die Lücke um ein Einfädeln unmöglich zu machen
	\item \enquote{the Punisher}, der Bestrafer: möchte jeden anderen Fahrer für gefühltes Fehlverhalten bestrafen, steigt ggf. aus dem Auto aus, um andere Fahrer direkt zu konfrontieren
	\item \enquote{the Philosopher}, der Philosophische: akzeptiert Fehlverhalten problemlos und versucht dieses rational zu erklären, versteht es seine Gefühle im Auto zu kontrollieren
	\item \enquote{the Avoider}, der Vermeider: behandelt Fahrer, die sich daneben benehmen, unpersönlich, tut sie als Gefahr ab
	\item \enquote{the Escapee}, der Flüchter: hört Musik oder spricht am Telefon, um sich zu isolieren, lenken sich mit ausgewählten sozialen Beziehungen ab, um nicht mit anderen Fahrern auf der Straße beschäftigen zu müssen, in erster Linie ist es aber eine Strategie, das Aufkommen von Frust zu vermeiden
\end{itemize}

Für zukünftige Entwicklungen der Simulation könnte man für diese oder ähnliche Fahrertypen Profile ausarbeiten und anlegen, die sich auf Abstands-, Beschleunigungs- und Verzögerungsverhalten sowie die Risikobereitschaft beim Aus- und Einscheren auswirken.
So könnte noch eine weitere Komponente mit Hinsicht auf Realitätsnähe geschaffen werden. 

Weiterhin könnte "Hupen" als Broadcast- oder Message-Ereignis festgelegt werden.
Dieses oder auch wiederholtes zu dichtes Auffahren und \enquote{geschnitten} werden, könnten sich auf eine Art Stresslevel auswirken, das als interner Belief im Agenten hinterlegt wird.
\\
Das Stresslevel wiederum könnte die Art und Weise, wie das Fahrzeug am Straßenverkehr teilnimmt, verändern. 
Ggf. auch in unterschiedlicher Art und Weise, je nach Fahrertyp.
\\
Über eine gewisse Zeit, baut sich erworbener Stress wieder ab.

Die Außenwahrnehmung eines Fahrzeuges könnte durch einen öffentlichen Belief gesteuert werden. 
Hier könnte z.B. der Hinweis, dass der Fahrer betrunken ist, andere Fahrzeuge vorsichtiger handeln lassen.








\subsection{NaSch in der Zukunft}

Wird as Nagel-Schreckenberg-Modell zukünftig noch von Bedeutung sein? 
Viele Entwicklungen im Automobilbereich gehen in Richtung hochassitentes oder gar autonomes Fahren.

bla bla bla Fehleinschätzungen können auch dem Rechner passieren bla bla bla

Sollten diese Systeme zu 100\% sicher funktionieren, kann man den Faktor Mensch, der meist die Unsicherheiten in das System Verkehr einbringt, aus der Gleichung entfernen.
Wichtig wäre dann aber auch die Kommunikation der Fahrzeuge untereinander.

Von diesem Punkt ist die Technik aktuell noch weit entfernt, siehe [www.sueddeutsche.de/auto/autonomes-fahren-der-entwicklungsaufwand-bei-selbstfahrenden-autos-ist-riesig-1.3838094], sodass der Mensch hinter dem Lenkrad wohl noch einige Zeit erhalten bleiben wird \dots und sei es als Backup bei Fehleinschätzungen der Technik.

https://www.welt.de/wirtschaft/article168550776/Fuer-autonomes-Fahren-gibt-es-ein-grosses-Hindernis.html
